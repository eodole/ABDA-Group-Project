\documentclass[12pt]{article}
\usepackage[utf8]{inputenc}
\usepackage{amsmath}
\usepackage{mathrsfs}
\usepackage{amssymb}
\usepackage{amsfonts}

\DeclareMathOperator{\logit}{logit}
\DeclareMathOperator*{\argmax}{arg\,max}
\DeclareMathOperator*{\argmin}{arg\,min}

\usepackage[
    a4paper,
    left = 2.5cm,
    right = 2.5cm,
    top = 2.5cm,
    bottom = 2.5cm
]{geometry}

\usepackage{graphicx}
\usepackage[headings]{fullpage}
\usepackage{hyperref}
\usepackage{fancyhdr}
 %For aligned formulas
 \usepackage{IEEEtrantools}
\usepackage{listings} %Source code listings https://en.wikibooks.org/wiki/LaTeX/Source_Code_Listings

% \lstset{
%     language=R, %You can always set to another language
%     breaklines,
%     deletekeywords={category},
%     basicstyle=\ttfamily\footnotesize,
%     otherkeywords={!,!=,~,\$,*,\&,\%/\%,\%*\%,\%\%,<-,<<-},
%     literate={~}{$\sim$}1 {<-}{{$\gets$}}1
% }

\usepackage[
    backend=biber,
    style=apa,
    maxcitenames=2,
    mincitenames=1,
    sorting=ynt
    ]{biblatex}
\addbibresource{bibfile.bib}

\setlength{\headheight}{15pt}
\pagestyle{fancy}
\fancyhf{}
\rhead{Your Names} % Fill in your name
%\lhead{Bayesian Statistics \& Probabilistic Machine Learning---Project Report}
\rfoot{Page \thepage}


\begin{document}

\begin{titlepage}
        \centering % Center all text
        \vspace*{\baselineskip} % White space at the top of the page
        
        {\huge YOUR TITLE}\\[0.2\baselineskip] % Title
        
        
        \vspace*{\baselineskip}
        
        {\Large --- Project Report ---\\
          Advanced Bayesian Data Analysis\\[\baselineskip]} % Tagline(s) or further description
        \vspace*{\baselineskip}
        
        {\LARGE Eldaleona Odole \and Leonor Cunha \and Anarghya Murthy\\[\baselineskip]} % Editor list  
        
        \vspace*{\baselineskip}

        \vfill
        
        \today \par % Location and year
        
        \vspace*{\baselineskip}

        {\itshape TU Dortmund University\par} % Editor affiliation
    \end{titlepage}

\clearpage

% \section{Template}
% This is a template for the project report in the course Advanced Bayesian Data Analysis.
% Fill in your title and names for the title page and header and follow the general structure. You don't have to use multiple chapters or sections at all for this short report, but if you do, don't go further than subsections--- \textcite{feynman1963lnphysics} didn't need to\ldots

% Some examples on how to use \LaTeX{} are shown below. If you want to reference something, you can do it as so:
% \textcite{mcelreath2016statistical}.

% \subsection{Images}
% \begin{figure}
%     \begin{center}
%     \includegraphics[width=0.5\textwidth]{figures/Normal_Distribution_PDF.pdf} %try to never force a figure placement
%     \caption{Probability density function for the Normal distribution. The red curve is the standard normal distribution. [By Inductiveload - self-made, Mathematica, Inkscape, Public Domain, \url{https://commons.wikimedia.org/w/index.php?curid=3817954}]}
%     \label{fig:normal_sample}
%     \end{center}
% \end{figure}

% This is an example for how to insert images into your document. When talking about a figure, you should always point out which one you mean, i.e., ``As you can see in Figure~\ref{fig:normal_sample}.''

% \subsection{Tables}
% A table has a caption \textit{above} the table as in Table~\ref{tab:my_label}.

% \begin{table} %You can place [h] immediately after \begin{table} to force the placement of the table. Generally speaking never do that---LaTeX usually places them in a sane way!
%     \centering
%     \caption{My caption.}
%     \label{tab:my_label}
%     \begin{tabular}{c|l} % l, r, and c justified inside cells.
%         \hline
%          Poisson & $\lambda$ \\ % Always end a line with \\
%          Normal &  $\mu$ and $\sigma$\\ 
%         \hline
%     \end{tabular}
% \end{table}

% \subsection{Formulas}
% This is a small example for how to include formulas into your document. $a^2 + b^2 = c^2$ will inline a formula, while
% $$c \leq a + b$$
% will give the formula its own line.

% \subsection{Formulas}
% You also might want to write out models:

% {\footnotesize % you align formulas using & 
% \begin{IEEEeqnarray*}{rCl}
% \mathrm{L}_i & \sim & \mathrm{Binomial}(n_i,p_i)\\
% \mathrm{logit}(p_i) & = & \alpha_{\mathrm{SUBJECT}[i]} + (\beta_P + \beta_{PC}C_i)P_i\\
% \alpha_{\mathrm{SUBJECT}} & \sim & \mathrm{Normal}(0,10) \\
% \beta_P & \sim & \mathrm{Normal}(0,10)\\
% \beta_{PC} & \sim & \mathrm{Normal}(0,10)
% \end{IEEEeqnarray*}
% }


% \subsection{Source code}
% Of course, formatting source code is always nice.
% \begin{lstlisting}
% m <- map(
%     alist(
%         height ~ dnorm(mu, sigma),
%         mu <- a + b*weight,
%         a ~ dnorm(0, 100),
%         b ~ dnorm(0, 10),
%         sigma ~ dunif(0, 50) 
%     ), 
%     data=d2)
% \end{lstlisting}

% \subsection{Math fonts}
% Different math fonts are also available to you:

% $\mathrm{ABCDE abcde 1234}$

% $\mathit{ABCDE abcde 1234}$

% $\mathnormal{ABCDEabcde1234}$

% $\mathcal{ABCDE abcde 1234}$

% $\mathscr{ABCDE abcde 1234}$

% $\mathfrak{ABCDE abcde 1234}$

% $\mathbb{ABCDE abcde 1234}$

\section{Introduction}

Every two years the United States elects 435 officials to the House of Representatives. The 435 House seats are allocated roughly proportional to population with additional constraint that each state must have at least one seat in the House of Representatives. However, despite being roughly equivalent in population size, the characteristics of voters living in each district varies greatly by district. 
As the U.S. is a two party system, the association between a voters demographics (gender, age, education etc.) and their propensity to vote for either a democratic or republican candidate is a topic of extensive study. However, little is known about the relationship between voting outcomes and the voters environment.  In a very general sense, conventional wisdom says that cities tend to be more “blue”, meaning voters in large cities tend to vote for democrats. Our project is focused on characterizing this relationship more concretely. 



\section{Dataset}

Our dataset was made by combining three independent datasets from the same time period. The first

\subsection{Data Cleaning}



\printbibliography

\end{document}
