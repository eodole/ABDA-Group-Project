\documentclass{beamer}
\usepackage[utf8]{inputenc}
\usepackage{dsfont}
\usetheme{Darmstadt}
\usecolortheme{default}

\begin{document}



%---
%This block of code defines the information to appear in the Title page
\title[About Beamer]{Does Urbanisation Predict Election Outcomes?}

\subtitle{A Bayesian's Perspective}

\author[Odole, Cunha, Murthy] % (optional)
{Eldaleona ~Odole \and Leonor ~Cunha \and Anarghya ~Murthy }

\institute[Technische Universität Dortmund] % (optional)
{
  
  Faculty of Statistics\\
}

\date{\today} % (optional)
% {Mid-Term Presentation}

\logo{\includegraphics[height=0.5cm]{tu.pdf}}

%End of title page configuration block
%---

%---
%The next block of commands puts the table of contents at the beginning of each section and highlights the current section:
% \AtBeginSection[]
% {
%   \begin{frame}
%     \frametitle{Table of Contents}
%     \tableofcontents[currentsection]
%   \end{frame}
% }
%---



%The next statement creates the title page.
\frame{\titlepage}

%---
%This block of code is for the table of contents after
%the title page
\begin{frame}
\frametitle{Table of Contents}
\tableofcontents
\end{frame}
%---

%---SECTION 1: INTRO
\section{Introduction}

\begin{frame}
\frametitle{Introduction}
\begin{itemize}
  \item \textbf{Research Question:}
  How does urbanization of a particular district affect result of an election in the US? 
  \item \textbf{Variable of Interest:}
  Winning party in the  House of Representatives 2022 General Election (binary)
\end{itemize}
\end{frame}



%slide
\section{Dataset Description}
\begin{frame}
\frametitle{Dataset}
We wanted to consider different factors in the analysis, with our primary focus being the urbanization of each House district. These factors included: 
\begin{enumerate}
  \item Demographic Data (US Census Bureau)
  \item Urbanization (FiveThirtyEight)
  \item Regional Information (US Census Bureau)
  \item Election Results (FiveThirtyEight)
\end{enumerate}
We combined different sources in order to create our data set containing 435 instances of 16 unique covariates.
\end{frame}  


\begin{frame}{Winning Party}
Our independent variable is Winning party in the 2022 Election.
 \includegraphics[width=0.9\textwidth]{plots/party_map.pdf}
\end{frame} 


\begin{frame}{Urban Index}
Our dependent variable of interest is the Urban Index from FiveThirtyEight.
 \includegraphics[width=0.9\textwidth]{plots/urbanindexmap.pdf}
\end{frame} 



\begin{frame}{Densities}
    \centering
    % First row
    \begin{minipage}{0.3\textwidth}
        \centering
        \includegraphics[width=\textwidth]{plots/urbanindex_density_plot.pdf}
    \end{minipage}
    \hfill
    \begin{minipage}{0.3\textwidth}
        \centering
        \includegraphics[width=\textwidth]{plots/Pct.Women_density_plot.pdf}
    \end{minipage}
    \hfill
    \begin{minipage}{0.3\textwidth}
        \centering
        \includegraphics[width=\textwidth]{plots/Mean.Income_density_plot.pdf}
    \end{minipage}
    
    \vspace{0.2cm} % Add vertical space between rows

    % Second row
    \begin{minipage}{0.3\textwidth}
        \centering
        \includegraphics[width=\textwidth]{plots/Pct.Retirees.65_density_plot.pdf}
    \end{minipage}
    \hfill
    \begin{minipage}{0.3\textwidth}
        \centering
        \includegraphics[width=\textwidth]{plots/Unempl.16plus_density_plot.pdf}
    \end{minipage}
    \hfill
    \begin{minipage}{0.3\textwidth}
        \centering
        \includegraphics[width=\textwidth]{plots/Median.Age_density_plot.pdf}
    \end{minipage}
    
    \vspace{0.2cm} % Add vertical space between rows

    % Third row
    \begin{minipage}{0.3\textwidth}
        \centering
        \includegraphics[width=\textwidth]{plots/Pct.Bsc.25plus_density_plot.pdf}
    \end{minipage}
    \hfill
    \begin{minipage}{0.3\textwidth}
        \centering
        \includegraphics[width=\textwidth]{plots/Total.Pop_density_plot.pdf}
    \end{minipage} 
    \hfill
    \begin{minipage}{0.3\textwidth}
        \hfill
    \end{minipage}
\end{frame}


\begin{frame}{Correlation Matrix}
    \includegraphics[width=0.9\textwidth]{plots/corrplot.pdf}
\end{frame}
%slide
% \begin{frame}
% \frametitle{Frame name}

% \begin{footnotesize}
% \begin{verbatim}
% # Set the maximum number of samples
% N <- 10000        
% # Pmf of x
% x <- 1        
% # Initialize a vector to store the final estimates
% estimates <- numeric(N)   
% # Write function to get estimate of the mean
% # for a given sample size 'n'

% get_est_mean <- function(lambda, n){
%   shelf <- numeric(n)    
%   for(i in 1:n){
%     x <- rpois(m, lambda[i])   #simulate i.i.d.
%                                 #Poisson for given lambda
%     shelf[i] <- mean(x)
%   }
%   return(mean(shelf))
% }

% \end{verbatim}
% \end{footnotesize}

% \end{frame}

\section{Model Setup}
%slide
\begin{frame}
  \frametitle{Model Assumptions}
  There are many people trying to predict US election outcomes, from the wealth of data available about voters. However we wanted to look at the voters in relation to their geography. In order to do this we assumed 
  \begin{itemize}
    \item District voting outcomes can be modeled via logistic regression 
    \item Districts are exchangable within each state and each state is exchangable within its region
    % would be nice to have a graphic showing the geographic hierachry 
    \item ??? 
  \end{itemize}
  \end{frame}

%slide
\begin{frame}
\frametitle{Model}
Let the response variable 'Winning Party' be \(y\), the predictor of interest 'Urban index' be \(x\), and the other covariates be a 15-dimensional vector \(z\). Let \(i\), \(j\), and \(k\) be the indices for the district, region, and state respectively. 

\[y_{i, j} \sim Ber.(logit^{-1}(\theta_{j}))\]

\[\theta_j := \beta_0, j + x_{i,j} * \beta_{1,j}  + z_{i, j}^T * \gamma_{1,j}\]

\end{frame}

\section{Priors}

%slide - prior place holder 
\begin{frame}
  \frametitle{Factor name}
  % what is it 

  % prior distribution equation 

  % explaination 
  \end{frame}

%slide - prior place holder 
\begin{frame}
  \frametitle{Urbanization Index}
  % what is it 
  Urbanization Index is a measure created by fivethirtyeight calculated based on populatin density of a given district. % should we have a source here? 
  %prior distribution graph 
  
  % prior distribution equation 

  % explaination 
  \end{frame}


\section{Results - Pooled Model}

%slide
\begin{frame}
\frametitle{=----}

\end{frame}


\section{Results - Unpooled Model}

%slide
\begin{frame}
\frametitle{=----}

\end{frame}

\section{Results - Hierarchical Model}

%slide
\begin{frame}
\frametitle{Varying Intercept Model}

\begin{center}
    \includegraphics[width=0.8\textwidth]{placeholder_hist_varying_intercept_1_level.jpeg}
\end{center}

\end{frame}


\begin{frame}
  \frametitle{Varying Intercept Model - II}%% wasn't rendering properly 
  
  % \begin{table} \centering
  %   \caption{Descriptive Statistics by Region} 
  %   \label{table:stats}
  %   \resizebox{\textwidth}{!}{ % Resizing the table to fit the frame width
  %   \begin{tabular}{@{\extracolsep{5pt}} cccccccccccc} 
  %   \toprule
  %   & Region & Variable & Mean & Median & SD & MAD & Q5 & Q95 & Rhat & ESS\_Bulk & ESS\_Tail \\ 
  %   \midrule
  %   1 & $1$ & r\_region\_index & $-0.661$ & $-0.627$ & $0.486$ & $0.405$ & $-1.484$ & $0.023$ & $1.014$ & $221.604$ & $240.881$ \\ 
  %   2 & $2$ & r\_region\_index & $0.272$ & $0.273$ & $0.496$ & $0.388$ & $-0.443$ & $1.035$ & $1.010$ & $249.762$ & $277.608$ \\ 
  %   3 & $3$ & r\_region\_index & $0.424$ & $0.410$ & $0.513$ & $0.417$ & $-0.414$ & $1.242$ & $1.004$ & $267.517$ & $238.103$ \\ 
  %   4 & $4$ & r\_region\_index & $-0.164$ & $-0.138$ & $0.495$ & $0.419$ & $-1.008$ & $0.534$ & $1.014$ & $262.794$ & $271.583$ \\ 
  %   \bottomrule
  %   \end{tabular}
  %   } % End of resizebox
  % \end{table}
\end{frame}



\section{Results - Varying Intercept Model}

%slide
\begin{frame}
\frametitle{=----}

\end{frame}

\section{Centred vs Non-centred Parameterisation/ Sampling Investigation with STAN}

%slide
\begin{frame}
\frametitle{STAN Code}

\end{frame}

\section{Raw References}

\begin{frame}{Raw references}
    \begin{itemize}
        \item stargazer
        \item tidybayes
        \item brms, stan
    \end{itemize}
\end{frame}




\end{document}
